\documentclass{exam}
\usepackage{amsmath, amsfonts}
\usepackage{verbatim}
\usepackage{graphicx}
\usepackage[super]{nth}

\DeclareMathOperator*{\argmin}{argmin}

\usepackage[hyperfootnotes=false]{hyperref}

\usepackage[usenames,dvipsnames]{color}
\newcommand{\note}[1]{
	\noindent~\\
	\vspace{0.25cm}
	\fcolorbox{Red}{Orange}{\parbox{0.99\textwidth}{#1\\}}
	%{\parbox{0.99\textwidth}{#1\\}}
	\vspace{0.25cm}
}


%\input{../macros}
%\renewcommand{\hide}[1]{#1}

\qformat{\thequestion. \textbf{\thequestiontitle}\hfill}
\bonusqformat{\thequestion. \textbf{\thequestiontitle}\hfill}

\pagestyle{headandfoot}

%%%%%% MODIFY FOR EACH SHEET!!!! %%%%%%
\newcommand{\duedate}{02.12.2020 (18:00)}
\newcommand{\due}{{\bf This assignment is due on \duedate.} }
\firstpageheader
{Due: \duedate}
{{\bf\lecture}\\ \assignment{1}}
{\lectors\\ \semester}

\runningheader
{Due: \duedate}
{\assignment{1}}
{\semester}
%%%%%% MODIFY FOR EACH SHEET!!!! %%%%%%

\firstpagefooter
{}
{\thepage}
{}

\runningfooter
{}
{\thepage}
{}

\headrule
\pointsinrightmargin
\bracketedpoints
\marginpointname{pt.}


\begin{document}

\section*{Exercise 7: Policy Gradient}
GitHub classroom: \url{https://classroom.github.com/a/i0PTF2Hw}

\noindent 
The purpose of this exercise is to get you accustomed to implementing policy gradient methods. For this, you will be implementing the REINFORCE algorithm to solve the \texttt{CartPole-v1} environment. Your tasks are the following:

\begin{questions}
	\titledquestion{Policy Gradient Implementation}
	\indent
	\begin{itemize}
		\item Complete the \texttt{Policy} class in the code with 2 Linear units to map the states to probabilities over actions.
		\item Implement \texttt{compute\_returns} method to compute the discounted returns $G_t$ for each state in a trajectory.
		\item Implement the \texttt{policy\_improvement} step to update the policy given the rewards and probabilities from the last trajectory.
		\item Use the policy in the \texttt{act} method to sample action and return its log probability.
	\end{itemize}	
	\titledquestion{Questions}
	\indent
	\begin{itemize}
		\item How does the length of the trajectories affect the training?
		\item How could a baseline be implemented to stabilize the training?
		\item Does the same network architecture and learning rate work for \texttt{LunarLander-v2}?
		\item How is the sample complexity (how many steps it takes to solve the environment) of this algorithm related to the DQN from the last exercise?
	\end{itemize}

\end{questions}

\noindent 
Please write your answers in \texttt{answers.txt}

\end{document}
